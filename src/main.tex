\documentclass[a4paper,12pt]{article}
\usepackage[utf8]{inputenc}
\usepackage[T1]{fontenc}
\usepackage[lithuanian]{babel} 
\usepackage{geometry}
\usepackage{amsmath}
\geometry{left=2.5cm, right=2.5cm, top=2.5cm, bottom=2.5cm}

\begin{document}

% Titulinis puslapis
\begin{titlepage}
    \centering
    {\large Vilniaus universitetas \\ Matematikos ir informatikos fakultetas \\ Informatikos institutas \\ Informatikos katedra \par}
    \vspace{2cm}
    {\LARGE \textbf{Semestro atsiskaitymo darbas} \par}
    \vspace{1cm}
    {\LARGE \textbf{Kompiuterių architektūros palyginimas: AN/UYK-43 ir Motorola 68HC11} \par}
    {\large \text{(Computer architecture comparison: AN/UYK-43 and Motorola 68HC11)} \par}
    \vspace{3cm}
    {\large 2 kurso 2 grupės studentas \par}
    \vspace{0.5cm}
    {\large \textbf{Autorius:} Dženardas Jevič \par}
    {\large \textbf{Vadovas:} Dr. Saulius Gražulis \par}
    \vfill
    {\large Vilnius \\ 2024 \par}
\end{titlepage}

\section{Įvadas.}
Šiame darbe bus nagrinėjamos dvi kompiuterių architektūros: AN/UYK-43 ir Motorola 68HC11, kurios atspindi skirtingas technologines kryptis ir taikymo sritis.\\
\textbf{AN/UYK-43} – tai CISC architektūros kompiuteris, sukurtas JAV kariniams jūrų laivynams. Pristatytas 1980-aisiais, jis buvo naudojamas kaip pagrindinė valdymo ir duomenų apdorojimo sistema laivų bei povandeninių laivų įrangoje. AN/UYK-43 buvo žinomas dėl savo patikimumo, gebėjimo apdoroti didelius duomenų kiekius ir palaikyti įvairius adresavimo režimus, įskaitant kintamo ilgio simbolių valdymą, kuris buvo naudingas teksto ir signalų apdorojimui.\\
\textbf{Motorola 68HC11} – tai populiarus 8 bitų mikrovaldiklis, sukurtas 1984 metais. Jis buvo plačiai naudojamas pramonėje, automobilių valdymo sistemose ir švietime dėl savo lankstumo, paprastumo ir efektyvaus periferinių įrenginių valdymo. 68HC11 architektūra pasižymėjo CISC instrukcijų rinkiniu, nedideliu energijos suvartojimu ir galimybe valdyti atminties bei įvesties/išvesties operacijas mažose įterptosiose sistemose.

\section{Kompiuterio bazė.}
\subsection{AN/UYK-43 (1960-ųjų pabaiga - 1970-ųjų pradžia).}
AN/UYK-43 kompiuterio elementinė bazė buvo sudaryta iš puslaidininkinių mikroschemų, tranzistorių logikos (TTL) ir feritinės magnetinių šerdžių atminties (32-bit Computers, Chapter 55, 2024). Visi komponentai atitiko karinius standartus (MIL-STD), užtikrinant patikimumą ir atsparumą ekstremalioms sąlygoms.
\subsection{Motorola 68HC11 (1980-ųjų pradžia).}
Motorola 68HC11 buvo sukurtas naudojant tranzistorius ir integrinius grandynus (IC). Tai buvo modernus mikroprocesorius, kurio elementai buvo pagrįsti tranzistoriais ir kitomis pažangiomis technologijomis (Wikipedia, 2024).

\section{Fizinės savybės.}
\subsection{Dydis ir svoris:}
\textbf{AN/UYK-43.} Kompiuteris buvo labai didelis. AN/UYK-43 buvo naudojamas kariuomenės, jūrų laivynų ir kitose srityse, todėl užėmė daug vietos ir buvo sunkus. Tikslūs matmenys ir svoris: 1,8 metrų aukščio, 0,5 metrų pločio, 0,57 metrų gylio, sveria nuo 666-756 kg., priklausomai nuo pasirinktos konfigūracijos (HP-9020C/AN/UYK-43 Study, 1987, p. 14). Ši technologija buvo naudojama kompiuteriuose, kurie buvo tokio dydžio, kad juos reikėjo montuoti į didelius spintų tipo korpusus.\\
\textbf{Motorola 68HC11.} Motorola 68HC11 buvo daug mažesnis ir lengvesnis nei AN/UYK-43. Jo dydis buvo mažas, ir jis galėjo būti naudojamas įvairiuose įrenginiuose, tokiuose kaip ekranų, spausdintuvų, klaviatūrų, modemų, įkrovimo kortelių telefonų ir buitinės technikos, pavyzdžiui, šaldytuvų, skalbimo mašinų ir mikrobangų krosnelių, valdikliai (Introduction to Motorola 68HC11, p. 5). Jis buvo mikroschemų dydžio, tai buvo mikroprocesorius, galintis tilpti į labai mažus fizinius įrenginius.

\section{Nuorodos:}
\begin{itemize}
    \item VIP CLUB (2024) 32-bit Computers, Chapter 55. Galima rasti adresu: https://vipclubmn.org/cp32bit.html (Accessed: 18 December 2024).
    \item Wikipedia (2024) Motorola 68HC11. Galima rasti adresu: https://en.wikipedia.org/wiki/Motorola_68HC11 (Accessed: 18 December 2024).
    \item Systems Exploration, Inc. (1987) HP-9020C/AN/UYK-43 Study. Galima rasti adresu: https://apps.dtic.mil/sti/tr/pdf/ADA188056.pdf (Accessed: 18 December 2024).
    \item -, (2000) Introduction to Motoola 68HC11. Galima rasti adresu: https://www.slideshare.net/slideshow/motorola-68hc11/22644023#5 (Accessed: 18 December 2024).
\end{itemsize}

\end{document}
